\documentclass{article}
\usepackage{amsmath, amssymb, amsthm, graphicx, enumerate, siunitx, textgreek, multicol}
\usepackage{hyperref}
\hypersetup{
    colorlinks=true,
    linkcolor=black,
    urlcolor=blue,
}

\newtheorem{theorem}{Theorem}%[section]
\newtheorem{lemma}[theorem]{Lemma}
\newtheorem{proposition}[theorem]{Proposition}
\newtheorem{corollary}[theorem]{Corollary}
\newtheorem{axiom}{Axiom}
\newtheorem*{remark}{Remark}
\newtheorem*{definition}{Definition}

\newcommand{\tablesection}[2] {
    \subsection{#1}  % name of the table
    #2  % description of the table
    \subsubsection{Attributes}
}

\newcommand{\attribute}[4] {
    \begin{itemize}
        \item {#1}  % attribute name
            \begin{itemize}
                \item \textbf{description}: #2  % description of the attribute 
                \item \textbf{data type}: #3  % data type of the attribute
                \item \textbf{domain}: #4  % domain of the attribute
            \end{itemize}
    \end{itemize}
}

\setcounter{tocdepth}{2}  % limit the depth of the table of contents

\title{Magic: The Gathering Database Documentation}
\author{John Kuroda\\Akil Marshall\\Israel Trusdell}
\begin{document}
\maketitle
\newpage
\tableofcontents
\newpage
\section{Philosophy of Design}
write something about why we did the things we did.
% A CARD is the primary object this database is designed around. A CARD may be printed in one or more SETs. A SET can be uniquely identified by it's set_name or it's set_num. 
% A CARD can be in more than one set.
% CARDs are identified by their name and the SET they are printed in.
% FORMATs have a list of allowed SETs.
% CARDs can be Restricted in a FORMAT (limited to 1 copy) or Banned in a FORMAT.
% CARD is a superclass (generalization) of CREATURE, ARTIFACT, LAND, ENCHANTMENT, PLANESWALKER, SORCERY, INSTANT.
% A CARD can be in multiple subcategories, i.e., “Artifact Land”, “Artifact Creature”, or ``Enchantment Creature''.
% SETs contain CARDs.

\section{Tables}
% \subsection{FORMAT}
% FORMATs have names that identify them.
% A FORMAT is a game type with a set of rules and a list of allowed SETs.
% FORMATs also have a list of restricted cards and banned cards.  Entire sets are not necessarily illegal as a whole because cards in a set that is not allowed may show up in a set that is allowed.
% FORMATs have a predetermined deck size.
% FORMATs have a maximum number of players per game (can be $\infty$).
% \subsection{SET}
% A SET has a unique number, name, and the year it was released.
\tablesection{CARD}{
    The MTG wiki had the following to say about what a card is.
    \begin{quote}
        In Magic: The Gathering, a card is the standard component of the game. The word card usually refers to a Magic card with a Magic card front and a Magic card back, or to double-faced cards.
    \end{quote}

    The CARD table is reflective of the elements you will find on a magic card.
}
\attribute{card\_name}{The name of the card.}{String}{Any valid card name.}
\attribute{text}{Everything in the text area of the card.}{String}{Any valid card text.}
\attribute{type}{The type of the card (creature, artifact, etc).}{String}{Any valid magic card type.}
\attribute{power}{The card's power.}{Integer}{Any non-negative integer.}
\attribute{toughness}{The card's toughness.}{Integer}{Any non-negative integer.}
\attribute{loyalty}{The card's loyalty.}{Integer}{Any non-negative integer.}

\tablesection{SET}{
    The MTG wiki had the following to say about what a set is.
    \begin{quote}
        A set in Magic: The Gathering is a pool of cards released together and designed for the same play environment. Cards in a set can be obtained either randomly through booster packs, or in box sets that have a fixed selection of cards. An expansion symbol and, more recently, a three-character abbreviation is printed on each card to identify the set it belongs to.
    \end{quote}

}
\attribute{set\_code}{The alphanumeric code associated with a set.}{String}{Combinations of letters and digits.}
\attribute{set\_name}{The name of the set.}{String}{Any valid set name.}
\attribute{year}{The year the set was released.}{Integer}{Any valid year.}
\attribute{set\_type}{The type of set it is (core, expansion, etc).}{String}{Any valid set type.}

\tablesection{FORMAT}{
    The MTG wiki had the following to say about what a format is.
    \begin{quote}
        Formats are different modes in which the Magic: The Gathering collectible card game can be played. Each format provides rules for deck construction and gameplay.
    \end{quote}
}
\attribute{format\_name}{The name of the format.}{String}{Any valid format name.}
\attribute{min\_deck\_size}{The minimum number of cards allowed in a deck.}{Integer}{Any non-negative integer.}
\attribute{max\_deck\_size}{The maximum number of cards allowed in a deck.}{Integer.}{Any integer, negative integers are interpreted as infinity.}
\attribute{copies\_allowed}{The maximum number of copies of a card allowed in a deck.}{Integer}{Any non-negative integer.}

\tablesection{IS\_ALLOWED}{
    This table is the implementation of the many-to-many relationship between FORMAT and SET\@. A format may allow many sets and a set may be included in many formats.
}
\attribute{set\_code}{A foreign key from SET.}{String}{Combinations of letters and digits.}
\attribute{format\_name}{A foreign key from FORMAT.}{String}{Any valid format name.}

\tablesection{CONTAINS}{
    This table is the implementation of the many-to-many relationship between CARD and SET. A card may be included in may sets and A set may contain many cards.
}
\attribute{set\_code}{A foreign key from SET.}{String}{Combinations of letters and digits.}
\attribute{card\_name}{A foreign key from CARD.}{String}{Any valid card name.}
\attribute{rarity}{The rarity of the card (common, uncommon, etc).}{String}{Any valid magic card rarity.}

\tablesection{LIMITATION}{
    This table is the implementation of the many-to-many relationship between FORMAT and CARD. A format may limit many cards and a card may be limited by many formats.
}
\attribute{format\_name}{A foreign key from FORMAT.}{String}{Any valid format name.}
\attribute{card\_name}{A foreign key from CARD.}{String}{Any valid card name.}
\attribute{limitation\_type}{The way in which a card is limited (banned, restricted, etc).}{String}{Any valid limitation.}

\tablesection{COLOR}{
    MTG wiki had the following to say about color.
    \begin{quote}
        Color is a basic property of cards in Magic: The Gathering, forming the core of the game's mana system and overall strategy.
    \end{quote}
}
\attribute{card\_name}{A foreign key from CARD.}{String}{Any valid card name.}
\attribute{color}{The color a card is associated with, usually indicated by the physical color of the card.}{String}{Any valid magic card color.}

\tablesection{COLOR\_COST}{
    Due to the fact that converted\_cost depends on cost\_string which is not the primary key. This table solves that problem.
}
\attribute{card\_name}{A foreign key from CARD.}{String}{Any valid card name.}
\attribute{cost\_string}{An alphanumeric representation of a cards mana cost.}{String}{Strings over the alphabet $\sum=\{R, U, G, B, W, X, \phi\}$ where $\phi\in\mathbb{Z}_{>0}$ and each string that contains $\phi$ begins with $\phi$.}
\attribute{converted\_cost}{
    The sum over a cards mana cost.
    \begin{table}[h!] % {h:here, t:top, b:bottom, p:page} add ! to override
        \centering
        \caption{How to sum a cost\_string.}
        \label{tab:convertedcost}
        \begin{tabular}{cc}
            $\sum$ & value \\
            \hline
            $R$ & 1 \\
            $U$ & 1 \\
            $G$ & 1 \\
            $B$ & 1 \\
            $W$ & 1 \\
            $X$ & 0 \\
            $\phi$ & $\phi$ \\
        \end{tabular}
    \end{table}
    Each occurrence of a character in a cost\_string is summed according to the above table.
}{Integer}{Any non-negative integer.}

\tablesection{DOUBLE\_CARD}{
    This table allows us to describe double faced cards which are magic cards with two faces.
}
\attribute{side\_a}{A foreign key from CARD, specifically a card\_name.}{String}{Any valid card name.}
\attribute{side\_b}{A foreign key from CARD, specifically a card\_name.}{String}{Any valid card name.}
\attribute{set\_code}{A foreign key from SET.}{String}{Combinations of letters and digits.}


\tablesection{SUPERTYPE}{
    Magic cards may have one or more supertypes, this table implements that one-to-many relationship.
}
\attribute{card\_name}{A foreign key from CARD.}{String}{Any valid card name.}
\attribute{supertype}{The supertype of the card (legendary, snow, etc).}{String}{Any valid magic card subtype.}

\tablesection{TYPE}{
    Magic cards may have one or more types, this table implements that one-to-many relationship.
}
\attribute{card\_name}{A foreign key from CARD.}{String}{Any valid card name.}
\attribute{type}{The type of the card (creature, artifact, etc).}{String}{Any valid magic card type.}

\tablesection{SUBTYPE}{
    Magic cards may have zero or more subtypes, this table implements that one-to-many relationship.
}
\attribute{card\_name}{A foreign key from CARD.}{String}{Any valid card name.}
\attribute{subtype}{The subtype of the card (equipment, curse, etc).}{String}{Any valid magic card subtype.}

\tablesection{COLOR\_IDENTITY}{
    Each mana symbol that appears on a card is included within that cards color identity.
    Each card is associated with one or more colors.
}
\attribute{card\_name}{A foreign key from CARD.}{String}{Any valid card name.}
\attribute{red}{A flag to indicate the cards alignment with red.}{Boolean}{Any valid boolean.}
\attribute{blue}{A flag to indicate the cards alignment with blue.}{Boolean}{Any valid boolean.}
\attribute{green}{A flag to indicate the cards alignment with green.}{Boolean}{Any valid boolean.}
\attribute{white}{A flag to indicate the cards alignment with white.}{Boolean}{Any valid boolean.}
\attribute{black}{A flag to indicate the cards alignment with black.}{Boolean}{Any valid boolean.}

\section{Domain Descriptions of Certain Attributes}
In this section we describe in detail and give examples of the attributes domain for those attributes that we can reasonably do so.

\subsection{SET.set\_type}
\subsubsection{Domain Values}
\begin{multicols}{3}
    \begin{itemize}
            \item archenemy
            \item box
            \item commander
            \item core
            \item draft\_innovation
            \item duel\_deck
            \item expansion
            \item from\_the\_vault
            \item funny
            \item masterpiece
            \item masters
            \item memorabilia
            \item planechase
            \item premium\_deck
            \item promo
            \item spellbook
            \item starter
            \item token
            \item treasure\_chest
    \end{itemize}
\end{multicols}

\subsection{CONTAINS.rarity}
\subsubsection{Domain Values}
\begin{multicols}{3}
    \begin{itemize}
            \item common
            \item uncommon
            \item rare
            \item mythic
    \end{itemize}
\end{multicols}
% \section{Lists of Type Instances}
% \subsection{Supertype}
% \begin{multicols}{3}
%     \begin{itemize}
%         \item Basic
%         \item Host
%         \item Legendary
%         \item Snow
%         \item World
%     \end{itemize}
% \end{multicols}
% \subsection{Artifact}
% \begin{multicols}{3}
%     \begin{itemize}
%         \item Clue
%         \item Contraption
%         \item Equipment
%         \item Food
%         \item Fortification
%         \item Gold
%         \item Key
%         \item Treasure
%         \item Vehicle
%     \end{itemize}
% \end{multicols}
% \subsection{Creature}
% \begin{multicols}{3}
%     \begin{itemize}
%         \item Advisor
%         \item Aetherborn
%         \item Ally
%         \item Angel
%         \item Antelope
%         \item Ape
%         \item Archer
%         \item Archon
%         \item Army
%         \item Artificer
%         \item Assassin
%         \item Assembly-Worker
%         \item Atog
%         \item Aurochs
%         \item Avatar
%         \item Azra
%         \item Badger
%         \item Barbarian
%         \item Basilisk
%         \item Bat
%         \item Bear
%         \item Beast
%         \item Beeble
%         \item Berserker
%         \item Bird
%         \item Blinkmoth
%         \item Boar
%         \item Bringer
%         \item Brushwagg
%         \item Camarid
%         \item Camel
%         \item Caribou
%         \item Carrier
%         \item Cat
%         \item Centaur
%         \item Cephalid
%         \item Chimera
%         \item Citizen
%         \item Cleric
%         \item Cockatrice
%         \item Construct
%         \item Coward
%         \item Crab
%         \item Crocodile
%         \item Cyclops
%         \item Dauthi
%         \item Demigod
%         \item Demon
%         \item Deserter
%         \item Devil
%         \item Dinosaur
%         \item Djinn
%         \item Dragon
%         \item Drake
%         \item Dreadnought
%         \item Drone
%         \item Druid
%         \item Dryad
%         \item Dwarf
%         \item Efreet
%         \item Egg
%         \item Elder
%         \item Eldrazi
%         \item Elemental
%         \item Elephant
%         \item Elf
%         \item Elk
%         \item Eye
%         \item Faerie
%         \item Ferret
%         \item Fish
%         \item Flagbearer
%         \item Fox
%         \item Frog
%         \item Fungus
%         \item Gargoyle
%         \item Germ
%         \item Giant
%         \item Gnome
%         \item Goat
%         \item Goblin
%         \item God
%         \item Golem
%         \item Gorgon
%         \item Graveborn
%         \item Gremlin
%         \item Griffin
%         \item Hag
%         \item Harpy
%         \item Hellion
%         \item Hippo
%         \item Hippogriff
%         \item Homarid
%         \item Homunculus
%         \item Horror
%         \item Horse
%         \item Hound
%         \item Human
%         \item Hydra
%         \item Hyena
%         \item Illusion
%         \item Imp
%         \item Incarnation
%         \item Insect
%         \item Jackal
%         \item Jellyfish
%         \item Juggernaut
%         \item Kavu
%         \item Kirin
%         \item Kithkin
%         \item Knight
%         \item Kobold
%         \item Kor
%         \item Kraken
%         \item Lamia
%         \item Lammasu
%         \item Leech
%         \item Leviathan
%         \item Lhurgoyf
%         \item Licid
%         \item Lizard
%         \item Manticore
%         \item Masticore
%         \item Mercenary
%         \item Merfolk
%         \item Metathran
%         \item Minion
%         \item Minotaur
%         \item Mole
%         \item Monger
%         \item Mongoose
%         \item Monk
%         \item Monkey
%         \item Moonfolk
%         \item Mouse
%         \item Mutant
%         \item Myr
%         \item Mystic
%         \item Naga
%         \item Nautilus
%         \item Nephilim
%         \item Nightmare
%         \item Nightstalker
%         \item Ninja
%         \item Noble
%         \item Noggle
%         \item Nomad
%         \item Nymph
%         \item Octopus
%         \item Ogre
%         \item Ooze
%         \item Orb
%         \item Orc
%         \item Orgg
%         \item Ouphe
%         \item Ox
%         \item Oyster
%         \item Pangolin
%         \item Peasant
%         \item Pegasus
%         \item Pentavite
%         \item Pest
%         \item Phelddagrif
%         \item Phoenix
%         \item Pilot
%         \item Pincher
%         \item Pirate
%         \item Plant
%         \item Praetor
%         \item Prism
%         \item Processor
%         \item Rabbit
%         \item Rat
%         \item Rebel
%         \item Reflection
%         \item Rhino
%         \item Rigger
%         \item Rogue
%         \item Sable
%         \item Salamander
%         \item Samurai
%         \item Sand
%         \item Saproling
%         \item Satyr
%         \item Scarecrow
%         \item Scion
%         \item Scorpion
%         \item Scout
%         \item Sculpture
%         \item Serf
%         \item Serpent
%         \item Servo
%         \item Shade
%         \item Shaman
%         \item Shapeshifter
%         \item Sheep
%         \item Siren
%         \item Skeleton
%         \item Slith
%         \item Sliver
%         \item Slug
%         \item Snake
%         \item Soldier
%         \item Soltari
%         \item Spawn
%         \item Specter
%         \item Spellshaper
%         \item Sphinx
%         \item Spider
%         \item Spike
%         \item Spirit
%         \item Splinter
%         \item Sponge
%         \item Squid
%         \item Squirrel
%         \item Starfish
%         \item Surrakar
%         \item Survivor
%         \item Tentacle
%         \item Tetravite
%         \item Thalakos
%         \item Thopter
%         \item Thrull
%         \item Treefolk
%         \item Trilobite
%         \item Triskelavite
%         \item Troll
%         \item Turtle
%         \item Unicorn
%         \item Vampire
%         \item Vedalken
%         \item Viashino
%         \item Volver
%         \item Wall
%         \item Warlock
%         \item Warrior
%         \item Weird
%         \item Werewolf
%         \item Whale
%         \item Wizard
%         \item Wolf
%         \item Wolverine
%         \item Wombat
%         \item Worm
%         \item Wraith
%         \item Wurm
%         \item Yeti
%         \item Zombie
%         \item Zubera
%     \end{itemize}
% \end{multicols}
% \subsection{Enchantment}
% \begin{multicols}{3}
%     \begin{itemize}
%         \item Aura
%         \item Carouche
%         \item Curse
%         \item Saga
%         \item Shrine
%     \end{itemize}
% \end{multicols}
% \subsection{Land}
% \subsubsection{Basic}
% \begin{multicols}{3}
%     \begin{itemize}
%         \item Plains
%         \item Island
%         \item Swamp
%         \item Mountain
%         \item Forest
%     \end{itemize}
% \end{multicols}
% \subsubsection{Non-Basic}
% \begin{multicols}{3}
%     \begin{itemize}
%         \item Desert
%         \item Gate
%         \item Lair
%         \item Locus
%         \item Urza's
%         \item Mine
%         \item Power-Plant
%         \item Tower
%     \end{itemize}
% \end{multicols}
% \subsection{Planeswalker}
% \begin{multicols}{3}
%     \begin{itemize}
%         \item Ajani
%         \item Aminatou
%         \item Angrath
%         \item Arlinn
%         \item Ashiok
%         \item Bolas
%         \item Chandra
%         \item Dack
%         \item Daretti
%         \item Davriel
%         \item Domri
%         \item Dovin
%         \item Elspeth
%         \item Estrid
%         \item Freyalise
%         \item Garruk
%         \item Gideon
%         \item Huatli
%         \item Jace
%         \item Jaya
%         \item Karn
%         \item Kasmina
%         \item Kaya
%         \item Kiora
%         \item Koth
%         \item Liliana
%         \item Nahiri
%         \item Narset
%         \item Nissa
%         \item Nixilis
%         \item Oko
%         \item Ral
%         \item Rowan
%         \item Saheeli
%         \item Samut
%         \item Sarkhan
%         \item Serra
%         \item Sorin
%         \item Tamiyo
%         \item Teferi
%         \item Teyo
%         \item Tezzeret
%         \item Tibalt
%         \item Ugin
%         \item Venser
%         \item Vivien
%         \item Vraska
%         \item Will
%         \item Windgrace
%         \item Wrenn
%         \item Xenagos
%         \item Yanggu
%         \item Yanling
%     \end{itemize}
% \end{multicols}
% \subsection{Instant}
% \begin{multicols}{3}
%     \begin{itemize}
%         \item Adventure
%         \item Arcane
%         \item Trap
%     \end{itemize}
% \end{multicols}
% \subsection{Sorcery}
% \begin{multicols}{3}
%     \begin{itemize}
%         \item Adventure
%         \item Arcane
%     \end{itemize}
% \end{multicols}
\end{document}
